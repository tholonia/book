% Options for packages loaded elsewhere
\PassOptionsToPackage{unicode}{hyperref}
\PassOptionsToPackage{hyphens}{url}
%
\documentclass[
]{article}
\usepackage{amsmath,amssymb}
\usepackage{iftex}
\ifPDFTeX
  \usepackage[T1]{fontenc}
  \usepackage[utf8]{inputenc}
  \usepackage{textcomp} % provide euro and other symbols
\else % if luatex or xetex
  \usepackage{unicode-math} % this also loads fontspec
  \defaultfontfeatures{Scale=MatchLowercase}
  \defaultfontfeatures[\rmfamily]{Ligatures=TeX,Scale=1}
\fi
\usepackage{lmodern}
\ifPDFTeX\else
  % xetex/luatex font selection
\fi
% Use upquote if available, for straight quotes in verbatim environments
\IfFileExists{upquote.sty}{\usepackage{upquote}}{}
\IfFileExists{microtype.sty}{% use microtype if available
  \usepackage[]{microtype}
  \UseMicrotypeSet[protrusion]{basicmath} % disable protrusion for tt fonts
}{}
\makeatletter
\@ifundefined{KOMAClassName}{% if non-KOMA class
  \IfFileExists{parskip.sty}{%
    \usepackage{parskip}
  }{% else
    \setlength{\parindent}{0pt}
    \setlength{\parskip}{6pt plus 2pt minus 1pt}}
}{% if KOMA class
  \KOMAoptions{parskip=half}}
\makeatother
\usepackage{xcolor}
\usepackage{graphicx}
\makeatletter
\def\maxwidth{\ifdim\Gin@nat@width>\linewidth\linewidth\else\Gin@nat@width\fi}
\def\maxheight{\ifdim\Gin@nat@height>\textheight\textheight\else\Gin@nat@height\fi}
\makeatother
% Scale images if necessary, so that they will not overflow the page
% margins by default, and it is still possible to overwrite the defaults
% using explicit options in \includegraphics[width, height, ...]{}
\setkeys{Gin}{width=\maxwidth,height=\maxheight,keepaspectratio}
% Set default figure placement to htbp
\makeatletter
\def\fps@figure{htbp}
\makeatother
\setlength{\emergencystretch}{3em} % prevent overfull lines
\providecommand{\tightlist}{%
  \setlength{\itemsep}{0pt}\setlength{\parskip}{0pt}}
\setcounter{secnumdepth}{-\maxdimen} % remove section numbering
\ifLuaTeX
  \usepackage{selnolig}  % disable illegal ligatures
\fi
\IfFileExists{bookmark.sty}{\usepackage{bookmark}}{\usepackage{hyperref}}
\IfFileExists{xurl.sty}{\usepackage{xurl}}{} % add URL line breaks if available
\urlstyle{same}
\hypersetup{
  hidelinks,
  pdfcreator={LaTeX via pandoc}}

\author{}
\date{}

\begin{document}

\section{Part I}\label{part-i}

\section{1: CHAOS}\label{1-chaos}

\subparagraph{\texorpdfstring{\textbf{Everything begins and ends in
chaos.}}{Everything begins and ends in chaos.}}\label{everything-begins-and-ends-in-chaos}

\subparagraph{\texorpdfstring{\textbf{Synopsis:} In the fundamental
trinity of reality, that of chaos, energy, and order, chaos comes first.
By definition, it exists before any \emph{thing} exists. It is the
engine of creation and annihilation and the always-present foundation of
all existence. Chaos exists across a spectrum that stretches between the
0 of nothingness and the ∞ of somethingness. This is the 1st duality of
creation from which all that ever was, is, or will emerge.
}{Synopsis: In the fundamental trinity of reality, that of chaos, energy, and order, chaos comes first. By definition, it exists before any thing exists. It is the engine of creation and annihilation and the always-present foundation of all existence. Chaos exists across a spectrum that stretches between the 0 of nothingness and the ∞ of somethingness. This is the 1st duality of creation from which all that ever was, is, or will emerge. }}\label{synopsis-in-the-fundamental-trinity-of-reality-that-of-chaos-energy-and-order-chaos-comes-first--by-definition-it-exists-before-any-thing-exists--it-is-the-engine-of-creation-and-annihilation-and-the-always-present-foundation-of-all-existence--chaos-exists-across-a-spectrum-that-stretches-between-the-0-of-nothingness-and-the--of-somethingness--this-is-the-1st-duality-of-creation-from-which-all-that-ever-was-is-or-will-emerge}

In the fundamental triad of reality - chaos, energy, and order - chaos
comes first. It exists by definition before any objects or things exist.
Chaos is the driving force behind creation and destruction and is always
present as the foundation of all existence. It exists along a spectrum
that ranges from the 0 of nothingness to the infinity of somethingness.
This is the first duality of creation from which everything that has
ever existed, exists, or will exist emerges."

\subparagraph{\texorpdfstring{\textbf{Keywords:} chaos, randomness,
pattern, duality, creation, zero,
infinity}{Keywords: chaos, randomness, pattern, duality, creation, zero, infinity}}\label{keywords-chaos-randomness-pattern-duality-creation-zero-infinity}

The Oxford English Dictionary gives three definitions of the word
\emph{chaos}:

\begin{itemize}
\item
  The formless matter supposed to have existed before the creation of
  the Universe.
\item
  The formless and disordered state of matter before the creation of the
  cosmos.
\item
  A state of extreme confusion and disorder.
\end{itemize}

I am not sure what the difference is between the first two definitions.
Regardless, we'll skip over the apparent contradiction of ``formless
matter'' existing before creation\ldots{} for now.

Some colloquial, older definitions of chaos describe it as matter before
there was any energy, but as we now know that matter itself is nothing
but energy, that description no longer applies.

There is also the Ancient Greek mythological definition of the word
χάος, \emph{Khaos}, referring to the void state preceding the creation
of the Universe or cosmos and personified in the Greek creation myths as
the god who was ``\emph{the first created being, from which came the
primeval deities Gaia, Tartarus, Erebus, and Nyx.''}

In physics, chaos is defined as ``\emph{The property of a complex system
whose behavior is so unpredictable as to appear random, owing to great
sensitivity to small changes in conditions}'', a. k. a. \emph{The
Butterfly Effect}, but as science writer James Gleick points
out:\footnote{Gleick, J. (1998). ``\textbf{Making a New Science}''.}

\begin{quote}
No one {[}of the chaos scientists he interviewed{]} could quite agree on
{[}a definition of{]} the word itself, and so instead gives descriptions
from a number of practitioners in the field. Those scientists that do
have a definition are hardly in agreement, as Philip Holmes defined
``chaotic'' as ``The complicated aperiodic attracting orbits of certain,
usually low-dimensional dynamical systems.''
\end{quote}

\ldots{} while Chinese theoretical physicist Bai-Lin Hao describes chaos
as ``a kind of order without periodicity''.

Another term for \emph{chaos} comes from cosmology, and that is the
\emph{primal void.}

Modern cosmology's Big Bang theory is quite compatible with this idea
that before there was anything, there was nothing but a primal void, a
term often used by cosmologists. The primal void is a descriptive
synonym for at least one state of chaos.

The primal void has also been called the \emph{gap between heaven and
Earth}, the \emph{abyss}, \emph{absolute-limitless nothing}, \emph{ex
nihilo}, \emph{primordial waters}, and several other terms, all trying
to convey the idea of an immeasurable nothingness, a concept that may
well be impossible for the mind to grasp. This nothingness also includes
(rather, excludes) the concept of time, as there can be no time in a
void of nothingness, at least according to Stephen Hawking.\footnote{``\textbf{The
  Beginning of Time}.'' Stephen Hawking,
  \url{https://www.hawking.org.uk/the-beginning-of-time.html}.}

Everyone from scientists to mystics seems to agree on the rather obvious
premise that before there was something, there was nothing, and then,
presto, somethingness appeared in the nothingness. What that
somethingness was is still unsettled. It could have been some form of
singularity that may or may not have spontaneously appeared and expanded
to create the Universe or some form of energy intense enough to warp
space to the point where it became mass or even some type of
multidimensional projection. In all these theories, and all beliefs and
speculations, there is the understanding that before there was
something, there was nothing.

\subsection{Nothingness}\label{nothingness}

How ``big'' was this nothingness? Is there a limit to nothingness? If
there is, how is that defined? If there is no limit, then is nothingness
infinite? These may sound like silly questions, but they are actively
being researched and debated in the worlds of science, philosophy, and
mathematics\footnote{Sorensen, Roy. ``\textbf{Nothingness.}'' Stanford
  Encyclopedia of Philosophy. Stanford University, February 28, 2022.
  \url{https://plato.stanford.edu/entries/nothingness/}.}.

First, we must be clear about what we mean by \emph{nothing} because
conceptually, there are numerous types of nothingness (and many types of
∞ as well, possibly infinite).

According to various authorities, there are multiple levels of nothing,
but a good overall understanding is provided by physicist \emph{Sabine
Hossenfelder}\footnote{Hossenfelder, Sabine. ``\textbf{What Is
  `Nothing'?}'' Backreaction, September 24, 2022.
  \url{https://backreaction.blogspot.com/2022/09/what-is-nothing.html}.},
who describes up to 9 levels of nothing. Below, we list these 9 levels,
their 3 general categories, and the 2 basics concepts of relative and
absolute nothingness that we will use in this book.

What level would ``I have nothing in my bank account'' be categorized?
Having a measurable balance of 0 puts your absolutely empty bank account
in Level 6, and being a measure of the non-physical entity of credit
puts it in level 7. Although being flat broke may feel like absolute
nothingness, finances are a value that is derived from material things
and is the transference of energy between things (people, businesses,
etc.), and could even be considered a ``field'', making your poverty a
relative form of nothingness.

\emph{Relative nothingness} is contextual. \emph{Absolute nothingness}
has no context because nothing exists to give it context, so there is
nothing that can be said about absolute nothingness. Even calling it
\emph{absolute nothingness} is limiting, as the concept of
\emph{absolute} implies ``\emph{to the largest degree possible}'', which
implies size, measure, and time, none of which exist in \emph{absolute
nothingness}, and just adding words like
\emph{super-duper-ultra-unlimited} doesn't fix this. Perhaps the only
thing that can be said about true nothingness is that there is nothing
we can say about it. In this way, primal, true, or absolute nothingness
is synonymous with the \emph{Tao} as described by Lao Tzu when he said:

\begin{quote}
The Tao that can be described \\
is not the eternal Tao.

The name that can be spoken is not the eternal Name. \\
The nameless is the boundary of Heaven and Earth.

The named is the mother of creation.
\end{quote}

Which is to say:

\begin{quote}
The nothingness that can be described \\
is not the eternal Nothingness.

The nothing that can be spoken of is not the eternal Nothing. \\
The eternal Nothingness is the boundary of all that can ever be and all
that is.

Nothing is the mother of creation.
\end{quote}

If there are (at least) 2 kinds of nothingness, does that mean there are
(at least) 2 kinds of dualities? Yes. Each of the 9 levels listed exists
in some type of duality (except perhaps \#9) and has a duality of some
sort, but with only 2 types of nothingness, we refer to only 2 types of
duality; the \emph{primal Duality}, which is the duality of the first
\emph{thing} that existed in the \emph{primal nothingness}, and
\emph{relative duality}, which is every duality that exists because of,
or within, the \emph{Primal Duality}.

We symbolize this primal duality as 1 and 0; from 1 and 0, all other
numbers appear. It might seem more sensible that the primal duality
would be that of ∞ and 0, which is also true, as ∞ is a concept
representing the totality or unity of all numbers and is not a number
itself, while the number 1 \emph{does} represent unity, and is even
defined as ``unity''. In this case, which is limited to the context of
\emph{unity}, 1=∞, or rather, the unity of 1 represents the unity of
all, and the mapping the range of 0→1 to 0→∞ is common in mathematics
and computer programming for exactly this reason.

All the dualities that follow the primal duality exist in the realm of
\emph{relative nothingness}. We have names for these relative dualities,
such as positive and negative, yin and yang, or any difference that
separates \emph{somethingness}. Measurement of any kind is a byproduct
of relative nothingness. This \emph{relative duality} is the duality of
our reality, the mundane duality that allows \emph{things} to exist,
energy to move, and has states of balance and imbalance. In this
context, balance refers to the balance \emph{within} movement, and not
the balance of stillness. Dancing, electricity, nature, the cosmos,
these are all examples of balance within movement.

Just as nothing must precede something, 0 precedes 1, so we can say that
everything starts with 0. We will be using the concept of 0 and
nothingness quite a bit, so let's review this ``simple'' and common
concept we use daily.

\subsubsection{Zero}\label{zero}

The general concept of nothingness is represented by the number 0, but
it was only recently in human history that we could even grasp the
concept of nothingness, and even then, its adoption took thousands of
years and was hotly debated. It is officially recognized that 0 was
first recorded in Mesopotamia around 3 B.C.. Still, the ancient Indian
\emph{Bakshali manuscript} from 1500 B.C. appears to use the symbol of a
dot (•) to represent 0. In either case, these were followed by 0's
appearance in Mayan Meso-America circa 4 A.D., again in Mayan
Meso-America in the 5th century, Cambodia in the 7th century, and China
and the Islamic countries in the 8th century. The concept of 0 didn't
reach Western Europe until the 12th century, and even then, it was not
entirely accepted as a legitimate number concept for hundreds of years,
which is a bit embarrassing considering that bees\footnote{Howard,
  Scarlett R., Aurore Avarguès-Weber, Jair E. Garcia, Andrew D.
  Greentree, and Adrian G. Dyer. ``\textbf{Numerical Ordering of Zero in
  Honey Bees.}'' \emph{Science} 360, no. 6393 (2018): 1124--26.
  \url{https://doi.org/10.1126/science.aar4975}.}, monkeys
\footnote{Sulkowski, G. ``\textbf{Can Rhesus Monkeys Spontaneously
  Subtract?}'' Cognition 79, no. 3 (2001): 239--62.
  \url{https://doi.org/10.1016/s0010-0277(00)00112-8}.},\footnote{Tsutsumi,
  Sayaka, Tomokazu Ushitani, and Kazuo Fujita. ``\textbf{Arithmetic-like
  Reasoning in Wild Vervet Monkeys: A Demonstration of Cost-Benefit
  Calculation in Foraging.}'' International Journal of Zoology 2011
  (2011): 1--11. \url{https://doi.org/10.1155/2011/806589}.},\footnote{Biro,
  Dora, and Tetsuro Matsuzawa. ``\textbf{Use of Numerical Symbols by the
  Chimpanzee (Pan Troglodytes): Cardinals, Ordinals, and the
  Introduction of Zero.}'' Animal Cognition 4, no. 3-4 (2001): 193--99.
  \url{https://doi.org/10.1007/s100710100086}.}, and crows\footnote{Kirschhock,
  Maximilian E.; Ditz, Helen M.; Nieder, Andreas, ``\textbf{Behavioral
  and Neuronal Representation of Numerosity Zero in the Crow}'' Journal
  of Neuroscience 2 June 2021, 41 (22) 4889-4896; DOI:
  \url{https://doi.org/10.1523/JNEUROSCI.0090-21.2021},
  \url{https://www.jneurosci.org/content/41/22/4889}} have no problem
with the concept of 0.

Similar to Dr. Elisabeth Kübler-Ross' well-known 5 stages of grief;
denial, anger, bargaining, depression, acceptance, the adoption of the
number 0, according to animal physiologist Andreas Nieder, went through
its own 4 stages\footnote{Nieder, Andreas. ``\textbf{Representing
  Something out of Nothing: The Dawning of Zero.}'' Trends in Cognitive
  Sciences 20, no. 11 (2016): 830--42.
  \url{https://doi.org/10.1016/j.tics.2016.08.008},
  \url{https://homepages.uni-tuebingen.de/andreas.nieder/Nieder\%20(2016)\%20TICS.pdf}},
which seems appropriate as the concept of nothing, on an existential
level, it not unlike death.

\begin{itemize}
\item
  \textbf{Stage 1:} Recognition of the absence of something.
\item
  \textbf{Stage 2:} Recognition of nothing vs. something (5th c. BC,
  Greece).
\item
  \textbf{Stage 3:} Recognition of 0 preceding 1 (7th c. AD, India ).
\item
  \textbf{Stage 4:} Ability to assign rules and properties to a symbolic
  representations (13th c. AD, North Africa).
\end{itemize}

\emph{Note: these dates are best-guess estimates based on documented
references.}

Curiously, humans achieved such an advanced state of intelligence yet
never developed the concept of 0. It's almost as if humans were gifted
the benefits of intelligence via genetic mutation, alien
intervention\footnote{As hypothesized by Dr. Immanuel Velikovsky, Erich
  von Däniken, and biblical scholar Zecharia Sitchin.}, or psychoactive
stimulation\footnote{As hypothesized by entheogenist Terrence McKenna
  and ethnopharmacologist Dennis McKenna in their ``Stoned Ape'' theory
  of evolutionary leaps.}, thereby skipping a critical phase of the
long, hard, slow work of evolution to get there.

It is not a coincidence that the modern world and the concept of 0
arrived simultaneously, as it was 0 that allowed for the creation of
things like modern math and calculus. This ability of 0 began to eclipse
the philosophical objections that you can't get something from nothing,
much to the chagrin of the Christian church, which was not very happy
that this handy number was being introduced by the Muslims, not to
mention that if something can come from nothing, what does that say
about god? \footnote{Seife, Charles. ``\textbf{Zero: The Biography of a
  Dangerous Idea}''. London: Souvenir Press, 2019.}.

\subsubsection{The Complexity of
Nothing}\label{the-complexity-of-nothing}

0 is challenging enough without the added confusion that comes from the
clearly ridiculous yet mathematically correct answer to 00=1 that many
calculators provide (including Google's calculator. The more realistic
answer is \emph{00=undefined}). However, X0 is always 0 as long as X!=0.
Although this may look like math mumbo-jumbo, it's pretty straight
forward: X0=X1-1=\(\frac{X^1}{X^1}\)=\(\frac{X}{X}\)=1. In addition to 0
representing nothing, this proof shows us how 0 also represents a state
of nothing that \emph{results from something}, as 1-1=0, or conceptually
speaking, \emph{something-something=nothing}. Hence, 0 is not only the
``womb'' from which all numbers emerge but also where numbers go when
they disappear from context.

I wonder if this is what Lao Tzu was also referring to in the \emph{Tao
Te Ching} when he said:

\begin{quote}
Yet mystery and reality \\
emerge from the same source. \\
This source is called darkness. \\
Darkness born from darkness. \\
The beginning of all understanding.
\end{quote}

Which is to say (in this case):

\begin{quote}
Yet the unknown and the known \\
emerge from the same source. \\
This source is called nothingness. \\
Somethingness born from nothingness. \\
The beginning of all understanding.
\end{quote}

This single number of 0 represents the antithesis of all other numbers,
just as the idea that all that ever did, does, or will exist, does so
because of the zen-like oxymoronic paradox of the ``existence of
nothingness''. Countless conjectures and proofs have been written on 0;
that it is equal to 1, or to the sum of all numbers, or to ∞, or
\(\frac{1}{\infty}\) and many other, or even all, values. This is not
hyperbole, as these concepts are hotly debated in mathematics and
philosophy. On its surface, the statement 0=∞ looks like a senseless
statement because 0 is a number but ∞ is a concept, so they can't ever
be equal. However, 0 is both a concept \emph{and} a number, as it has a
\emph{quantitative} value of 0 and a \emph{qualitative} value of
\emph{nothingness}. This is similar to how the number 1 has a
\emph{quantitative} value of 1 and a \emph{qualitative} value of
\emph{unity}. Of course, all numbers are concepts, and most common
numbers have well-known qualitative, but the first pair of
\emph{nothingness} and \emph{unity} started it all. It's worth noting
that Parmenides, Leibniz, and other giants in the history of math equate
\emph{unity} with ∞ with the argument along the lines of all that
\emph{is} collectively defines the unity of existence; hence, the
concept of 1=∞, which leads to concepts like
\(\frac{1}{\infty}\)=1∴1-∞=0∴∞-1=0∴0+1=∞, etc., etc. While these
concepts may not work so well quantitatively as when balancing your
budget, they are fundamental qualitative concepts that have been
examined by great minds, from Heraclitus to Hegel.

The truth is, or rather \emph{a} truth is, the numbers 0 and 1 are
remarkably flexible as they are two concepts that encapsulate the idea
of the nothing of nothingness and the totality of somethingness; the two
poles that define the arena or spectrum wherein all things exist. What
happens inside this arena? In a word: \emph{order}. There is no order in
the \emph{nothing of nothingness} for there is nothing to order, and
there is no order in the totality of somethingness as there is no form,
structure, sequence, etc. If 0 is the mathematical concept of
nothingness, then 1 is the ultimate mathematical black hole, and in that
sense, 1 \emph{does} represent ∞. 0 and 1 (or 0 and ∞ if you prefer) are
two states of chaos. Order only exists \emph{between} these 2 states,
and our understanding, discovery, and invention of order allow us to
know this.

\paragraph{\texorpdfstring{\textbf{Key 1:} chaos is a state lacking any
order, time, or energy; total nothingness;
0.}{Key 1: chaos is a state lacking any order, time, or energy; total nothingness; 0.}}\label{key-1-chaos-is-a-state-lacking-any-order-time-or-energy--total-nothingness--0}

\paragraph{\texorpdfstring{\textbf{Key 2:} chaos is a state of total
energy and matter; total somethingness;
∞}{Key 2: chaos is a state of total energy and matter; total somethingness; ∞}}\label{key-2-chaos-is-a-state-of-total-energy-and-matter-total-somethingness}

\subsection{Order and the chaos of 0 and
∞}\label{order-and-the-chaos-of-0-and}

It was Newton's contemplation of \(\frac{0}{0}\) that led to his
invention of calculus, and while we are taught that equations with 0 or
∞ are problematic and best to stay away from, calculus can prove that
∞∞, 1∞, 00, \(\frac{\infty}{\infty}\), 0×∞, ∞-∞ all equal
\(\frac{0}{0}\), and yet, even today, we are not sure what
\(\frac{0}{0}\) actually is. We can see evidence of this in current
research, such as the research of the Chief Editor of the publication
\emph{Causation}, Ilija Barukčić, titled ``\emph{Zero Divided by Zero
Equals One}''\footnote{Barukčić, Ilija. ``\textbf{Zero Divided by Zero
  Equals One.}'' Journal of Applied Mathematics and Physics 06, no. 04
  (2018): 836--53. \url{https://doi.org/10.4236/jamp.2018.64072}.},
which starts with:

\begin{quote}
Objective: Accumulating evidence \textbf{indicates} that zero divided by
zero equals 1
\end{quote}

And concludes with:

\begin{quote}
Conclusion: The findings of this study \textbf{suggest} that zero
divided by zero equals one.
\end{quote}

Or, in a paper co-authored by Ilija Barukčić that appeared in
\emph{Journal of Applied Mathematics and Physics} \footnote{Barukčić, J.
  and Barukčić, I. (2016) ``\textbf{Anti Aristotle---The Division of
  Zero by Zero}''. Journal of Applied Mathematics and Physics, 4,
  749-761. doi: 10.4236/jamp.2016.44085.}:

\begin{quote}
A solution of the philosophically, logically, mathematically and
physically far reaching problem of the division of zero by zero (0/0) is
still not in sight.
\end{quote}

In addition to this ambiguity, not only can 0 = ∞, it can equal any
number. That is actually a true math statement (I didn't just make it
up). In math, when an answer can be many values, it is called an
\emph{indeterminate} answer, meaning an equation, like \(\frac{0}{0}\),
has no single or fixed value that can be determined (even though it
looks likes the answer should be 1). The same can be said for
\(\frac{\infty}{\infty},\frac{\infty}{0}\) ,0×∞, 00, ∞0, 1∞, ∞=∞, as
they are all \emph{indeterminate}. Indeterminism is the equivalent of
mathematical chaos, which supports our claim that 0 and ∞ are
qualitative representations of chaos. They are also the only numbers
that represent a concept that has no value; 0 being the explicit lack of
any value, and ∞ which is a concept that represents the opposite of 0
and yet can have many different forms. In general, and how it is used in
this book, ∞ is meant as a frame for all positive numbers, as in \emph{0
\textless{} positive\_numbers \textless{} ∞.}

While the above is a mathematical argument for why 0 and \(\infty\)
represent states of chaos, we can also use the current definitions of
chaos to make the same argument. While we can say that nothingness has
no order or periodicity and is therefore chaotic, \emph{chaos} also has
another definition that means precisely the opposite, similar to the
\emph{∞=0} concept. Take, for example, the study of chaos in the origins
of the Universe that was undertaken by Chicago's Northwestern University
physicist Adilson Motter, who concluded that 10-36 seconds after the Big
Bang happened; there was a state of \emph{total chaos}. Do you know what
was happening at 10-36 seconds after the Big Bang? Everything, and in a
temperature of over 1 trillion degrees. Motter concluded from his
study:\footnote{Northwestern University. ``\textbf{Big bang was followed
  by chaos, mathematical analysis shows}''. ScienceDaily. ScienceDaily,
  8 September 2010.
  \url{www.sciencedaily.com/releases/2010/09/100907171642.htm}}

\begin{quote}
``Now we establish once and for all that {[}the universe{]} is
chaotic.''
\end{quote}

We are not claiming here that the Big Bang theory is correct, but we are
asking the question; How can the concept of \emph{chaos} describe the
state of the void of total nothingness as well as the state of all
matter and energy in the Universe, total somethingness?

The definition of chaos that works for both is \emph{``the degree that
order is present in any state''}. Both extreme states of total
nothingness and total somethingness have no order, pattern, or
periodicity, which is quite compatible with another common definition of
\emph{chaos} as synonymous with \emph{unpredictability}. A more flexible
definition of \emph{chaos} comes from the authors of
``\emph{Introduction to Complex Systems, Sustainability, and
Innovations}'',\footnote{Thomas, Ciza, et al. ``\textbf{Introduction to
  Complex Systems, Sustainability and Innovation.}'' \emph{Complex
  Systems, Sustainability and Innovation}, 2016, doi:10.5772/66453.},
which simply states:

\begin{quote}
chaos explores the transitions between order and disorder. An order
arises from the ever growing disorder of the Universe - chaos and order
together.
\end{quote}

But this is still too vague for our purposes because it is more
reasonable to consider \emph{chaos} as a spectrum from 0 to ∞. More
correctly, the spectrum of \emph{chaos} is the inverse of the spectrum
of \emph{order} (which is further defined later on), because just as
there is no such \emph{thing} as darkness, only a lack of the
\emph{thing} that is is light, chaos is the lack of order. For this
reason, we initially define using a slightly inaccurate definition shown
below. However, we will be refining this concept as we continue.

\paragraph{\texorpdfstring{\textbf{Key 3:} chaos as a measure of
pattern, order, and
predictability.}{Key 3: chaos as a measure of pattern, order, and predictability.}}\label{key-3-chaos-as-a-measure-of-pattern-order-and-predictability}

Chaos is not the same as randomness. A random event is
\emph{non-deterministic}, meaning it can't be determined when it will
happen because it has no pattern, rule, or reason (that we can discover)
and no known immediate cause. At least, that is how we define the word
as it is used in this book. It may be the case that the concept of
randomness simply exists to label unexplainable events that are actually
chaotic in nature but far beyond our current abilities to understand,
like a super-chaotic event. An example of this is from the website
RANDOM.ORG, which generates random numbers. In order to make them more
truly random, they use atmospheric noise as an input. That ambient noise
is not random but is chaotic. Another example is how computer random
number generators use electrical noise and heat as inputs to their
number generator. Again, these are chaotic inputs.

Another way to understand the difference between randomness and chaos is
in the value of π (pi). The value of π is infinite in numbers and the
sequence of those numbers is quite predictable using simple math, which
is how we can calculate the value of π, but there is no pattern to the
numbers, so there is no way to determine the following number based on
the previous numbers. This is similar to how prime numbers can only be
calculated as they have no pattern. This makes the value of π chaotic
because the value is deterministic yet produces a series of numbers that
appear random and uniformly distributed (a.k.a. \emph{normal}
distribution). We say ``appear'' because there is no way to prove that π
is random or \emph{normal}, just as there is no way to prove that the
infinite numerical sequence contains every possible combination of
numbers that can ever exist, but evidence suggests that's
\emph{probably} the case\footnote{Bailey, David H., Jonathan M. Borwein,
  Cristian S. Calude, Michael J. Dinneen, Monica Dumitrescu, and Alex
  Yee. ``\textbf{An Empirical Approach to the Normality of π.}''
  Experimental Mathematics 21, no. 4 (2012): 375--84.
  \url{https://doi.org/10.1080/10586458.2012.665333},
  \url{https://carmamaths.org/resources/jon/normality.pdf}}. So, chaos
has a pattern, but it is unpredictable, and \textbf{randomness has no
pattern and therefore contains \emph{every} pattern}. Here we see an
instance of the concept of not only 0=∞, but that ∞ exist \emph{within}
0.

\begin{quote}
There's a beauty to Pi that keeps us looking at it... the digits of Pi
are extremely random. They have no pattern, and in mathematics that's
really the same as saying they have every pattern.''
\textbf{\textasciitilde Jonathan Borwein, mathematician}
\end{quote}

Depending on who you ask, you can get a number of different definitions
of ``random'', but the definition we will be referring to is:

\paragraph{\texorpdfstring{\textbf{Key 4:} A random event is a
spontaneous event with no apparent
cause.}{Key 4: A random event is a spontaneous event with no apparent cause.}}\label{key-4-a-random-event-is-a-spontaneous-event-with-no-apparent-cause}

Of course, this leaves a lot of room for speculation as to the
randomness of an event, as there are undoubtedly many cases where we
just can't see the cause. What appears as random could easily be related
to the \emph{butterfly effect}, which states that small changes can have
a growing and cascading effect. This idea is often exemplified in the
question, ``\emph{Does the flap of a butterfly's wings in Brazil set off
a tornado in Texas?}'' which was also the title of a 1972
talk\footnote{Lorenz, Edward N. ``\textbf{The Essence of Chaos}''.
  Seattle: Univ. of Washington Press, 2008. Appendix 1 ``\textbf{The
  Butterfly Effect}'',
  \url{http://climate.envsci.rutgers.edu/climdyn2017/LorenzButterfly.pdf}.
  Lorenz was the founder of chaos theory which began with his 1963 paper
  ``\textbf{Deterministic Nonperiodic Flow}'',
  \url{https://journals.ametsoc.org/view/journals/atsc/20/2/1520-0469_1963_020_0130_dnf_2_0_co_2.xml}}
given by Edward Norton Lorenz, a mathematician, meteorologist, and
founder of chaos theory.

While this is a bit of an aside, it is a fascinating piece of history
and an excellent example of the \emph{butterfly effect} in so far as how
it can affect society and how a candid meeting of two strangers in a
ball in Prague in 1896 would be responsible for World War I and World
War II.

\includegraphics{/home/jw/books/tholonia/chapters/Images/sophie.png}The
two strangers were Arch Duke Franz Ferdinand and Sophie Chotek, a
duchess and the daughter of a Bohemian Count, who met, fell in love, and
got married. The problem was Sophia, being a mere duchess was not
royalty, and given royalty's strict adherence to their self-aggrandizing
customs, it was forbidden that she appear next to the Archbishop in any
official royal ceremonies. The Arch Duke may have loved Sophie, but he
was still an obedient autocrat, and so he followed the rules. However,
this meant he \emph{was} allowed to have her by his side during
non-royal ceremonies, such as when he was acting as the
Inspector-General of the Austria-Hungarian Army. Taking advantage of
this loophole, he decided to show off his wife to the world by taking a
public trip to inspect the Bosnian army with his wife by his side. To
ensure everyone saw them together, they traveled in an open car for all
to see. It was during this public demonstration of his undying love that
the Serbian nationalist, Gavrilo Princip, ran up to the car and shot
both of them at point-blank range, killing them instantly.

Austria was outraged, demanding an apology from Serbia. Serbia, while
denouncing the assassinations, refused to apologize, stating they had
nothing to do with the plot. Austria responded by declaring war on
Serbia, which forced treaty-bound Russia to ally with Serbia. Germany,
who was treaty-bound to Austria, declared war on Russia, causing France
and Great Britain to come to Russia's aid. World War I devastated
Germany, which laid the fertile ground for the rise of nationalism and
Hitler, which resulted in World War II.

All of the death and destruction of these wars was a consequence of a
chance meeting at a ball, plus countless other \emph{butterfly effects},
such as Gavrilo stopping to buy a sandwich which happened to place him
at the right time and place, and every other interaction that has ever
taken place since the First Cause started the show. But such stories are
deceptive because they take away from the inconceivably profound
realization that every trivial act is also a result of everything that
has happened since the beginning of time. Yes, it's quite possible that
WWI would have occurred in any case, but in this timeline of history,
this is what \emph{did} happen, just as you may not have spilled your
drink if you hadn't turned your head because you heard a weird sound,
but you did because everything that has ever happened since the
beginning of time was leading up to that moment. This may sound like
there is no such thing as ``free will,'' but we won't get into that now
as it is addressed towards the end of the book.

One of the classic ``truly random'' events in the Universe is
radioactive decay, but only the decay of a single atom. As a group, the
decay is entirely predictable, which is how we calculate half-lives of
radioactive material. This is the same as saying we know that there is a
1 in 366 chance of getting into a car accident on a 1,000-mile car trip,
but it is impossible to predict the who, when, or where of any one
particular accident. Insurance companies try to minimize the odds with
reams of data on the probabilities of accidents regarding teen drivers,
aggressive drivers, weather, speeding, impaired driving, driver error,
distracted driving, etc., which narrows the range but still can't
predict one accident. Still, this does not make individual accidents
random. Suppose Sally gets drunk because she just got fired for
aggressive behavior, and she decides to drive home from the bar in a
snowstorm while arguing with her husband on the cell phone and going 30
miles an hour over the speed limit, and she gets into an accident with
Bob. In that case, there is nothing random about that, nor is there
anything random about how Bob's car was where it was when it got hit.
Both Sally and Bob were on a path resulting from a series of chaotic
events (i.e., life). Each event increased or decreased the probability
of an accident. In Sally's case, those events clearly created a high
probability scenario for an accident. Bob, on the other hand, being a
careful driver who was paying attention and was going slow, had a very
low probability of an accident, yet he got hit anyway. We would say this
was a random event (``bad luck'') for Bob but a highly predictable event
for Sally. In this scenario, ``random'' means an unpredictable yet
deterministic chaotic system (Bob's life) that influenced another
unpredictable yet deterministic chaotic system (Sally's life). Because
neither system is predictable, it is impossible to predict when or if
they will interact; thus, ``random'' is often synonymous with
``theoretically unpredictable,'' as this random accident was the effect
of a cause created by the intersection of 2 chaotic systems.

A more common example of this is the idea of a stock market. The price
of a stock is unpredictable, but it is not truly random, as the price is
determined by the individual actions of thousands of buyers and sellers,
each making non-random decisions based on their financial interests.
Each of these cause/effect chains is a system in and of itself, and all
these systems combined form the larger system of a stock market.

\includegraphics{/home/jw/books/tholonia/chapters/Images/L-sys-2.png}Chaos
is \emph{deterministic} because it adheres to rules and even has a
pattern, but the effects over time create unpredictable results. Chaos
does not happen in one moment. It happens over time, and what happens
next depends on what happened before, making chaos a self-similar or
self-referencing process. Because it is \emph{change} that happens over
time, there are two components to consider; a \emph{growth factor} and a
\emph{limiting factor}. For example, in the image to the right, we start
with a simple pattern that never changes but can replicate itself in its
children, and after 10 generations, it turns into a tree. The
\emph{growth factor} is self-generating, and the \emph{limiting factor}
is that all of the variables (length, angle, color) are permanently
fixed. Popular real-world chaotic systems are things like the weather,
economics, growth patterns, etc.. Still, in actuality, everything that
grows, moves, or has energy moving through it, has elements of chaos in
its system that are influenced by millions of variables forever
changing.

Below are some diagrams to help make this concept clearer.

\textbf{Row A:} The top-left shows the changes to a simple form where a
line half the length extends from its parent line at 20°. This newly
extended line will then also have a line half its length extending from
it, and so on. The top-right image is the same pattern, but with 90°
instead of 20°. They are side-by-side to show how the natural form of
plants is the same pattern as a now-common man-made grid pattern. Why is
that significant? Because this grid pattern was adopted as a way to grow
and expand new territory more efficiently, which is why the architects
of the New World adopted it in the 1700s. Thomas Jefferson first used
this pattern to ``gridify'' the entire country to make it easier to
expand into. City planners found the same pattern far more effective
than the more organically formed cities of old Europe. The major driving
force behind this pattern was financial, of course, and it's no surprise
that this efficient form of development appeared and grew in parallel
with the industrial revolution and the radical expansion of city
populations, such as London, which grew 600\% from 1775 to 1885. We see
this same evolution of patterns in biology, such as grid cells, which
are the cells that allow us to navigate our environment and which form a
triangular grid with their impulses. It may seem like a stretch to claim
that the industrial revolution, capitalism, and suburban roads are just
another instance of the same patterns that determine tissue growth and
the stripes of a Zebra Fish, but that is precisely the case\footnote{Heller,
  E., \& Fuchs, E. (2015). ``\textbf{Tissue patterning and cellular
  mechanics}``. \emph{The Journal of cell biology}, \emph{211}(2),
  219--231. \url{https://doi.org/10.1083/jcb.201506106},
  \url{https://www.ncbi.nlm.nih.gov/pmc/articles/PMC4621832/}}. The only
difference is the context in which these patterns are being applied and
the resources at hand.

\textbf{Row B:} The middle row shows the simplest of all forms, the
triangle. In this case, a new line is always created at the end of the
previous line but rotated left 120°. At the end of that child line, a
new parent line is created but rotated in the opposite direction (whose
child will also rotate right, and so on). In the left image, all
variables remain the same. In the middle image, the deviation was
allowed to wander 1° in either direction. In the right image, the
deviation was allowed to wander in any direction. These images represent
the three states of order; perfect and unchanging, transitioning, and
total chaos.

This is also an example of an instance of a relative duality. While the
prime duality is that of primal nothingness and primal somethingness, a
relative duality is that of a relative nothingness and a relative
somethingness. In this case, the relative somethingness is the lines
formed by the rule, and the relative nothingness is where there are no
lines. Together they create a perfect form (according to the rules); in
other words, an archetype. It is perfect because the rules are perfect.
When we add a tiny bit of imperfection of 1° (or 0.27\%) of randomness,
the form is no longer perfect but still identifiable. With 100\% of
randomness, it is total chaos, a perfect mess. This is an example of the
path from order to one type of chaos (the path from chaos to order is
addressed in the chapter on energy).

\textbf{Row C:} Because chaos is deterministic (but appears random),
whatever a chaotic system does, it will do exactly the same way every
time if all the variables remain the same. In our world, these variables
are dynamic, so there will always be some changes to some variables.
This is why all pine trees don't look identical. The row shows five
instances of the 5th generation of the same source pattern where slight
variations in length and divergence were allowed. This is meant to
demonstrate that while growth is a chaotic system, evolution is a
chaotic system with some amount of randomness in it, for without those
slight variations, nothing would ever change. Reality expands in chaos
but evolves in randomness. At least, that is the premise we start from
because random events can exist within the chaos of change, but the
chaos of change can not exist in a random event, and the Universe is
certainly a system of change.

When looking at these images, remember that these are only 2-dimensional
patterns of 10 or fewer generations from their starting point, with only
length and divergence changing slightly. In the 3-dimensional world we
live in, the starting point from which all form descends is the
beginning of existence. The variables and generations at any and every
moment since the beginning of time are countless, and the change in the
variables range from minuscule to dramatic. The impossibility of
accounting for all the details necessary to make a prediction of a
chaotic system is why it is unpredictable.

\end{document}
